%\documentclass[]{beamer}
%\documentclass[handout]{beamer}
%\usepackage{epsf}
\documentclass[pdf]{beamer}
\usepackage{amsfonts}
\usepackage{amsthm,amssymb,amsbsy}
%\usepackage{amscd}
\usepackage{amsmath}
\usepackage{xcolor}


\usepackage{mathtools}

\usepackage{array}
\setbeamercovered{transparent}
\newcounter{savedenum}
\newcommand*{\saveenum}{\setcounter{savedenum}{\theenumi}}
\newcommand*{\resume}{\setcounter{enumi}{\thesavedenum}}

\usepackage{romannum}

\usepackage[super]{nth}
%\nth{1}, \nth{2}, \nth{3}, \nth{4}

%\usepackage{tabularx}
\usepackage{verbatim}


\usepackage{lmodern}
\usepackage{animate}
\usepackage{graphics}
\usepackage{color}
\usepackage{caption}
\captionsetup{labelfont = {color = black},font = {it, scriptsize}}

\usepackage{bbm}


\usepackage{mathdots}% http://ctan.org/pkg/mathdots
\usepackage{yhmath}% http://ctan.org/pkg/yhmath

%\usepackage{algorithm2e}
\usepackage[linesnumbered,algoruled,boxed,lined]{algorithm2e}

\usepackage{multicol}

\usepackage{draftwatermark}
%\usepackage{graphicx}
\usepackage{lipsum}



\usepackage[latin1]{inputenc}
\usepackage{tikz}
\usepackage{tikz-dependency}
\usetikzlibrary{shapes,arrows}
\usetikzlibrary{positioning, decorations.markings}
\usetikzlibrary{calc}
\usetikzlibrary{positioning}

\newcommand{\tikzmark}[1]{\tikz[overlay,remember picture] \node (#1) {};}
\newcommand{\DrawBox}[1][]{%
    \tikz[overlay,remember picture]{
    \draw[red,#1]
      ($(left)+(-0.2em,0.9em)$) rectangle
      ($(right)+(0.2em,-0.3em)$);}
}

\usepackage{pgfplots}
\pgfplotsset{compat=1.6}

\pgfplotsset{soldot/.style={color=blue,only marks,mark=*}} \pgfplotsset{holdot/.style={color=blue,fill=white,only marks,mark=*}}


\usetheme{Darmstadt}
%\usetheme{AnnArbor}
%\usetheme{Antibes}
%\usetheme{Bergen}
%\usetheme{Berkeley}
%\usetheme{Berlin}
%\usetheme{Boadilla}
%\usetheme{boxes}
%\usetheme{CambridgeUS}
%\usetheme{Copenhagen}
%\usetheme{Darmstadt}
%\usetheme{default}
%\usetheme{Frankfurt}
%\usetheme{Goettingen}
%\usetheme{Hannover}
%\usetheme{Ilmenau}
%\usetheme{JuanLesPins}
%\usetheme{Luebeck}
%\usetheme{Madrid}
%\usetheme{Malmoe}
%\usetheme{Marburg}
%\usetheme{Montpellier}
%\usetheme{PaloAlto}
%\usetheme{Pittsburgh}
%\usetheme{Rochester}
%\usetheme{Singapore}
%\usetheme{Szeged}
%\usetheme{Warsaw}


%\useoutertheme{shadow}
\usecolortheme{dolphin}

%\usetheme{CambridgeUS}
%\usepackage[latin1]{inputenc}
%\usefonttheme{professionalfonts}
%\usepackage{times}
%\usepackage{tikz}
%\usepackage{amsmath}
%\usepackage{verbatim}
%\usetikzlibrary{arrows,shapes}



%---------------------------------------------------------

\theoremstyle{definition} \theoremstyle{plain} \theoremstyle{remark}

\newcommand{\supp}{\mathrm{supp}}
\newcommand{\IE}{\mathbb{E}}
\newcommand{\1}{\mathbf{1} }
\newcommand{\PP}{\mathbb{P}}
\newcommand{\IQ}{\mathbb{Q}}
\newcommand{\IN}{\mathbb{N}}
\newcommand{\IO}{\mathbb{O}}
\newcommand{\IA}{\mathbb{A}}
\newcommand{\IC}{\mathbb{C}}
\newcommand{\II}{\mathbb{I}}
\newcommand{\IF}{\mathbb{F}}
\newcommand{\RR}{\mathbb{R}}
\newcommand{\DD}{\mathbb{D}}
\newcommand{\HH}{\mathbb{H}}
\newcommand{\spn}{\mathrm{span}}
\newcommand{\cov}{\mathrm{cov}}
\newcommand{\HS}{\mathcal{L}_{\mathrm{HS}}}
\newcommand{\trace}{\mathrm{Tr}}
\newcommand{\LL}{\mathcal{L}}
\newcommand{\s}{\mathcal{S}}
\newcommand{\ee}{\mathcal{E}}
\newcommand{\ff}{\mathcal{F}}
\newcommand{\hh}{\mathcal{H}}
\newcommand{\bb}{\mathcal{B}}
\newcommand{\oo}{\mathcal{O}}
\newcommand{\dd}{\mathcal{D}}
\newcommand{\g}{\mathcal{G}}
\newcommand{\half}{\frac{1}{2}}
\newcommand{\T}{\mathcal{T}}
\def\IE{{\mathbb{E}}}
\def\F{{\cal F}}
\def\Cov{{ \mbox{Cov}}}
\def\Var{{ \mbox{Var}}}
\def\Tr{{\mbox{Tr}}}

\def\hyph{-\penalty0\hskip0pt\relax}

\newcommand{\indicator}{\mathbbm{1}}
\newcommand{\Ii}{\mathbbm{1}}

\newtheorem{proposition}[theorem]{Proposition}
\newtheorem{remark}[theorem]{Remark}
%\newtheorem{solution}[theorem]{Solution}
\newtheorem{summary}[theorem]{Summary}
\newtheorem*{Def}{Definition}
\newtheorem{thm}{Theorem}[section]
\newtheorem{lem}[thm]{Lemma}
\newtheorem{cor}[thm]{Corollary}
\newtheorem{prop}[thm]{Proposition}
\newtheorem{Prop}{Proposition}         %% un-numbered
\newtheorem*{sublemma}{Sublemma}
\newtheorem*{THM}{Theorem}

\usepackage{tikz}
\usetikzlibrary{shapes.geometric,arrows}
\tikzset{
  basics/.style={minimum width=20mm, minimum height=7.5mm, text centered, draw=black},
  startstop/.style={rectangle, rounded corners, basics, fill=red!30},
  io/.style={trapezium, trapezium left angle=70, trapezium right angle=110, basics, fill=blue!30},
  process/.style={rectangle, basics, fill=blue!30},
  decision/.style={ellipse,basics, fill=green!30},
  arrow/.style={thick,->,>=stealth},
}

%---------------------------------------------------------
\title{NYU Hospital Heart Transplant Progress Report}

\author{{Viraat Singh, Miao Wang, Runze Wang, Yiqing Liu, Allen Yang}\newline
{Professor: Ali Hirsa}
}



\graphicspath{}
%\graphicspath {{figures/}}

\date{June 26, 2024}

\begin{document}

\frame{\titlepage}

%============================================
\section{Tasks Accomplished}
%============================================

\subsection{}
\begin{frame}
\frametitle{Work Completed}
\begin{itemize}
    \item Extraction of Training Data
    \vspace{0.12in}
    \item Construction of PostgreSQL Database
    \vspace{0.12in}
    \item Model Construction (In Progress)
    \vspace{0.12in}
\end{itemize}
\end{frame}

\subsection{}
\begin{frame}
\frametitle{Data Extraction}
\begin{itemize}
    \item Developed a Python script to extract a balanced set of experimental data.
    \vspace{0.12in}
    \item Constructed dataset includes 400 accepted transplant samples, 400 rejected transplant samples, 100 conditionally accepted transplant samples, and 100 bypassed transplant samples.
    \vspace{0.12in}
\end{itemize}
\end{frame}

\subsection{}
\begin{frame}
\frametitle{Data Related Questions}
\begin{itemize}
    \item Is there a significant difference between "conditionally accept" and "accept," and between "reject" and "bypass"? This will determine whether we should treat them as four distinct categories.
    \vspace{0.12in}
    \item If our model were to make mistakes, which type would be more acceptable: rejecting patients who should be accepted or accepting patients who should be rejected?
\end{itemize}
\end{frame}

\subsection{}
\begin{frame}
\frametitle{Database Construction}
\begin{itemize}
    \item Integrated heart transplant-related tables using PostgreSQL, simplifying the data management extraction process.
    \vspace{0.12in}
    \item Saved scripts for database construction, enabling easy pipeline creation for other organs such as lungs and livers.
    \vspace{0.12in}
\end{itemize}
\end{frame}

\subsection{}
\begin{frame}
\frametitle{Model Construction}
\begin{itemize}
    \item Brainstormed and drafted initial functions for the model, subject to further refinement.
    \vspace{0.12in}
    \item Started researching and discussing relevant academic papers to inform the process.
    \vspace{0.12in}
    \item Ongoing progress ready for formal initiation with selected columns. We will have more to present on this next week.
    \vspace{0.12in}
\end{itemize}
\end{frame}

%===================================
\section{Looking Forward}
%===================================

\subsection{}
\begin{frame}
\frametitle{Looking Forward}
\begin{itemize}
    \item Formally constructing a prototype model.
    \vspace{0.12in}
    \item Developing a list of function explanations for professional review to ensure logical accuracy.
    \vspace{0.12in}
    \item Testing and training the model with different success parameters such as acceptance rate, misclassification rate, etc. (Additional ideas are welcome.)
    \vspace{0.12in}
\end{itemize}
\end{frame}

\end{document}
